\documentclass[dvipsnames]{beamer}

\usetheme[sectionpage=none,numbering=none]{metropolis}

\newenvironment{wideitemize}{\itemize\addtolength{\itemsep}{1em}}{\enditemize}
\newcommand{\sitem}{\item[\raisebox{.45ex}{\rule{.6ex}{.6ex}}]}
\renewcommand{\arraystretch}{1.5}

\title{Programming with Evidence}
\subtitle{\small The introduction to an introduction to Agda}
\author{Uma Zalakain\\\tiny Formal Methods group, University of Glasgow}
\date{}
\institute{Basque Center for Applied Mathematics\\\tiny November 23, 2021}

\begin{document}

\maketitle

\begin{frame}{About me}
\begin{wideitemize}
  \sitem second year PhD student at the Formal Methods group at University of Glasgow
  \sitem machine verification of typed process calculi:\\
    using proof assistants to model typed concurrency languages and to verify their meta-theory
  \sitem programming languages theory, concurrency theory,\\
    type theory, distributed systems
\end{wideitemize}
\end{frame}

\begin{frame}{Yesterday: Zermelo-Fraenkel Set Theory}
\begin{wideitemize}
  \sitem a foundation for mathematics
  \sitem untyped, $x \in A$ is a \emph{proposition}
  \sitem elements can belong to different sets
  \sitem a set is fully characterised by its elements
  \sitem primitives: set operations ($\cup$, $\cap$)
  \sitem predicate logic
  \sitem non-constructive
\end{wideitemize}
\end{frame}

\begin{frame}{Today: Martin-L\"of Type Theory}
\begin{wideitemize}
  \sitem also a foundation for mathematics
  \sitem typed, $x : A$ is a \emph{judgment}
  \sitem an element has a unique type
  \sitem  a type is \emph{not} characterised by its elements
  \sitem primitives: datatypes and functions
  \sitem propositions as types ($\Pi$ and $\Sigma$ model predicate logic)
  \sitem constructive: \emph{a programming language!}
\end{wideitemize}
\end{frame}

\begin{frame}{Propositions as Types}
\begin{wideitemize}
  \sitem propositions are (proof-relevant) types
  \sitem proofs are programs
  \sitem evidence is data
  \sitem constructivism: proofs have computational content,\\
    \emph{you can run them}
\end{wideitemize}
\end{frame}

\begin{frame}{Propositions as Types}
\center
\begin{tabular}{ c | c }
  proposition & type \\ \hline{}
  $\bot$ & $\mathtt{Zero}$ \\
  $\top$ & $\mathtt{One}$ \\
  $A \land B$ & $\mathtt{A \times B}$ \\
  $A \lor B$ & $\mathtt{A \uplus B}$ \\
  $A \implies B$ & $\mathtt{A \to B}$ \\
  $\neg A$ & $A \to \bot$ \\
  $\forall x . \, P \, x$ & $\mathtt{\Pi \, (x : A) \, (P \, x)}$ \\
  $\exists x . \, P \, x$ & $\mathtt{\Sigma \, (x : A) \, (P \, x)}$ \\
\end{tabular}
\end{frame}

\begin{frame}{Interactive Proof Assistants}
\begin{itemize}
  \sitem system checks proofs for correctness
  \sitem system helps the user to construct those proofs interactively
  \sitem interactive proving becomes interactive programming
  \sitem requires less trust, easier to refactor with confidence
  \sitem educational value --- instant feedback for the student
  \sitem easy to reuse: shared library of definitions and proofs
  \sitem proofs compute!
  \sitem a lot of fun!
\end{itemize}
\end{frame}

\begin{frame}{Interactive Proof Assistants}
  \begin{wideitemize}
  \sitem Coq: based on the Calculus of Inductive Constructions, heavy use of tactics
  \sitem Lean: based on the Calculus of Inductive Constructions, small kernel, support for quotient types
  \sitem Idris2: based on Quantitative Type Theory, supports linearity annotations, focuses on compilation
  \sitem Agda: very close to Martin-L\"of Type Theory, handles proof terms directly
  \end{wideitemize}
\end{frame}


\begin{frame}{Agda}
\begin{wideitemize}
  \sitem developed mainly at Chalmers, Sweden
  \sitem clean syntax, unicode support
  \sitem based on dependent pattern matching
  \sitem mostly used in:
    \begin{itemize}
      \sitem Programming Language Theory
      \sitem Category Theory
      \sitem Homotopy Theory
    \end{itemize}
\end{wideitemize}
\end{frame}

\begin{frame}{Dependent Types}
  \begin{wideitemize}
    \sitem types contain arbitrary expressions
    \sitem allow correct-by-construction problem modelling
    \sitem pre and post conditions can be tightened\\ using types as specifications 
    \sitem empty types rule out impossible cases
  \end{wideitemize}
\end{frame}

\begin{frame}{About the tutorials}
\begin{wideitemize}
  \sitem Monday to Thursday
  \sitem 9h total
  \sitem interactive --- an emacs buffer
  \sitem available online: \url{https://umazalakain.github.io/agda-bcam/}
  \sitem recorded for posterity (including mistakes)
\end{wideitemize}
\end{frame}

\begin{frame}{About the tutorials}
\begin{itemize}
  \sitem simple and composite types
  \sitem unicode and misfix operators
  \sitem interactive programming
  \sitem record types
  \sitem Curry-Howard correspondence
  \sitem dependent function types
  \sitem indexed data types
  \sitem parametrised modules
  \sitem with abstraction
  \sitem automated evidence-providing solvers
\end{itemize}
\end{frame}

\begin{frame}{Bibliography}
\begin{wideitemize}
  \sitem \emph{Introduction to Agda}, \textbf{Andreas Abel}, 8th Summer School on Formal Techniques (SSFT’18) Menlo College, California, US
  \sitem \emph{Computer Aided Formal Reasoning}, \textbf{Thorsten Altenkirch}, 2010
  \sitem \emph{A Practical Agda Tutorial}, \textbf{Péter Diviánszky and Ambrus Kaposi}, 2013
\end{wideitemize}
\end{frame}
\end{document}